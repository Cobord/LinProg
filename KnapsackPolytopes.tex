\documentclass[11pt]{article}
\usepackage[margin=1in]{geometry} 
\usepackage{url}
\usepackage{graphicx} %If you want to include postscript graphics
\DeclareGraphicsExtensions{.pdf,.eps,.fig,.pdf_tex}
\usepackage{float}

\usepackage[colorlinks,
             linkcolor=black!50!red,
             citecolor=blue,
             pdftitle={Knapsack Polytope Ehrhart Polynomial},
             pdfauthor={Ammar Husain},
             pdfsubject={Polytopes,Knapsack},
             pdfkeywords={Polytopes,Knapsack,Ehrhart},
             pdfproducer={pdfLaTeX},
             pdfpagemode=None,
             bookmarksopen=true
             bookmarksnumbered=true]{hyperref}

\geometry{letterpaper}   

\usepackage{amsmath}
\usepackage{amssymb,amsfonts,bbm,mathrsfs,stmaryrd}

%%% Theorems and references %%%
\usepackage[amsmath,thmmarks]{ntheorem}
\usepackage{hyperref}

\newcommand\abs[1]{ \mid #1 \mid }

\theoremstyle{change}

\newtheorem{defn}[equation]{Definition}
\newtheorem{definition}[equation]{Definition}
\newtheorem{thm}[equation]{Theorem}
\newtheorem{theorem}[equation]{Theorem}
\newtheorem{prop}[equation]{Proposition}
\newtheorem{proposition}[equation]{Proposition}
\newtheorem{lemma}[equation]{Lemma}
\newtheorem{cor}[equation]{Corollary}
\newtheorem{exercise}[equation]{Exercise}
\newtheorem{example}[equation]{Example}
\theorembodyfont{\upshape}
\theoremsymbol{\ensuremath{\Diamond}}
\newtheorem{eg}[equation]{Example}
\newtheorem{remark}[equation]{Remark}
\theoremstyle{nonumberplain}
\theoremsymbol{\ensuremath{\Box}}
\newtheorem{proof}{Proof}
\qedsymbol{\ensuremath{\Box}}

\numberwithin{equation}{section}
\usepackage{cleveref}

\usepackage{svg}
\usepackage{listings}
\usepackage{caption}
\DeclareCaptionFont{white}{\color{white}}
\DeclareCaptionFormat{listing}{%
  \parbox{\textwidth}{\colorbox{gray}{\parbox{\textwidth}{#1#2#3}}\vskip-4pt}}
\captionsetup[lstlisting]{format=listing,labelfont=white,textfont=white}
\lstset{frame=lrb,xleftmargin=\fboxsep,xrightmargin=-\fboxsep}

\title{Knapsack Polytope Ehrhart Polynomial}
\author{Ammar Husain}

\begin{document}

\section{Knapsack Problem}

\begin{definition}[Knapsack Problem]
Let $A_1 \cdots A_n$ be goods with values $v_i$ and weights $w_i$. The knapsack can include any natural number of each good with the condition that the total weight should be less than $W$. Call the numbers of each as $x_i$.

\begin{eqnarray*}
V &=& \sum v_i x_i\\
\sum w_i x_i &\leq& W\\
\end{eqnarray*}

The goal is to maximize $V$.

\end{definition}

\begin{definition}[Decision Problem Formulation]
Given a $V_0$, can we find a solution with $V \geq V_0$?
\end{definition}

\begin{lemma}[NP-Complete]
This decision problem is NP-complete. The optimization problem is NP-hard.
\end{lemma}

\begin{lemma}[Augmented Form]

The weight inequality can be rephrased by introducing another natural number $s$ to represent $W - \sum w_i x_i$. The system then becomes solving the following system in $\mathbb{N}^{n+2}$.

\begin{eqnarray*}
\begin{pmatrix}
1 & - v_1 & \cdots & -v_n & 0\\
0 & w_1 & \cdots & w_n & 1\\
\end{pmatrix}
\begin{pmatrix}
V\\
x_1\\
\cdots\\
x_n\\
s\\
\end{pmatrix}
&=&
\begin{pmatrix}
0\\
W\\
\end{pmatrix}
\end{eqnarray*}

Another way is given by

\begin{eqnarray*}
\begin{pmatrix}
v_1 & \cdots & v_n & 0\\
w_1 & \cdots & w_n & 1\\
\end{pmatrix}
\begin{pmatrix}
x_1\\
\cdots\\
x_n\\
s\\
\end{pmatrix}
&=&
\begin{pmatrix}
V\\
W\\
\end{pmatrix}
\end{eqnarray*}

The solution with largest $V$ is sought.

\end{lemma}

\begin{lemma}[0-1 Algorithm]

Let $m[i,j]$ be the maximum value attained by using items $\leq i$ and weight $\leq j$. In particular $i=0$ is the case when no items are allowed to be used and so no value can be attained no matter the weight.

\begin{lstlisting}[label=KnapsackDynamicProgramming,caption=KnapsackDynamicProgramming]
for j from 0 to W do:
    m[0, j] := 0

for i from 1 to n do:
    for j from 0 to W do:
        if w[i] > j then:
            m[i, j] := m[i-1, j] \% We can't use item i with weight restriction j
        else:
            m[i, j] := max(m[i-1, j], m[i-1, j-w[i]] + v[i]) \% We can use it so we must decide whether it is worth doing so.
\end{lstlisting}
\end{lemma}

\section{Ehrhart Polynomial}

\begin{definition}[Ehrhart Polynomial]
For a polytope $\mathcal{P} \subset \mathbb{R}^d$ with integral vertices, the count of points in $t\mathcal{P} \bigcap \mathbb{Z}^d$ as a function of positive integer $t$ denoted $L_\mathcal{P} (t)$.
\end{definition}

\begin{theorem}[Ehrhart]
$L_{\mathcal{P}} (t)$ is a polynomial of degree $d$. The leading term is the relative volume. $L_\mathcal{P} (0) = \chi (\mathcal{P})$ and $L_{\mathcal{P}} (-t) = (-1)^d L_{\mathcal{P}^o} (t)$ where $\chi$ is Euler characteristic and $\mathcal{P}^o$ denotes relative interior.
\end{theorem}

If the vertices are rational but not integral, we get a quasipolynomial instead.

\begin{definition}[Quasipolynomial]
An expression of the form $c_d (t) t^d + \cdots c_1 (t) t + c_0 (t)$ with the $c_i (t)$ periodic functions in $t$.
\end{definition}

\section{Ehrhart Polynomial of Linear Programs}

\begin{theorem}
Let $\mathcal{P}$ be the polytope given by $x \in \mathbb{R}_{\geq 0}^d$ and constraint $Ax = b$ where $A \in M_{m \times d}(\mathbb{Z})$ and $b \in \mathbb{Z}^m$ are integral. $\mathcal{P}$ has rational vertices, not necessarily integral. Then the Ehrhart quasipolynomial is given by

\begin{eqnarray*}
L_\mathcal{P} (t) &=& \frac{1}{(2\pi i)^m} \int_{\abs{z_1}=\epsilon_1} \cdots  \int_{\abs{z_m}=\epsilon_m} \frac{z_1^{-t b_1 -1} \cdots z_m^{-t b_m -1}}{(1-\bold{z}^{\bold{c_1}}) \cdots (1-\bold{z}^{\bold{c_d}})}\\
\end{eqnarray*}
where $\bold{z}^{c_j}$ stands for the product $z_1^{A_{1j}} \cdots z_m^{A_{mj}}$. That is $\bold{z}$ stands for $z_1$ through $z_m$ and $\bold{c_j}$ is the j'th column of $A$. Also $0 < \epsilon_1 \cdots \epsilon_m <1$ are distinct real numbers.

\end{theorem}

\begin{proof}
\url{https://arxiv.org/pdf/math/0202267.pdf}
\end{proof}

Looking at the second linear program phrasing of the knapsack problem givs $m=2$ and $d=n+1$ with $V=b_1$. $L_\mathcal{P} (1)$ counts the number of solutions and we wish to see how it depends on $b_1$. For some critical value, $V_{crit}$, the point count will be $0$ from then on. That means the optimal value that can be attained will be $V_{crit}-1$. That is the last case when the polytope has integral points.

\begin{eqnarray*}
L_\mathcal{P} (t) &=& \frac{-1}{4 \pi^2} \int_{\abs{z_1}=\epsilon_1} \int_{\abs{z_2}=\epsilon_2} \frac{z_1^{-t V -1} z_2^{-t W -1}}{(1-z_1^{v_1} z_2^{w_1}) \cdots (1-z_1^{v_n} z_2^{w_n}) (1-z_2)}\\
\end{eqnarray*}

\begin{cor}
Expanding the term corresponding to the n'th good in a geometric series gives an expansion of $L_\mathcal{P} (t)$ for the first $n-1$ goods with total value $V-kv_n$ and total value $W-kw_n$ summed over $k$. This is obviously interpreted as deciding how many of good $n$ to pack and packing the remaining knapsack with the other $n-1$ goods.
\end{cor}

\begin{lemma}
The other values of $t$ get interpreted as what happens if you are allowed to break each item into $\frac{1}{t}$ parts and then pack those into the knapsack. For large $t$, this tends to the continuous linear programming problem where one is allowed to use arbitrary fractions of the goods.
\end{lemma}

\begin{example}

\end{example}

\end{document}